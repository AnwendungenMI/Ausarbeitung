\chapter{Risiko und Anforderungsanalyse}
Dieser Bereich umfasst die Beschreibung der Risiken und eine entsprechende Risikoplanung, in deren Anschluss die die Funktionalen und Nichtfunktionalen Anforderungen definiert werden.\\
Um während der Entwicklung und Umsetzung des Projekts nicht von projektgefährdenden Risiken überrascht zu werden, wurde vor der Entwicklung eine Risikoanalyse durchgeführt und entsprechende Maßnahmen geplant. Hierfür wurden die einzelnen Komponenten des Projekts separat analysiert und die auftretenden Risiken identifiziert. Die benötigte/ verwendete Hardware dieses Projekts lässt sich in die folgenden drei Komponenten unterteilen:
\\
\begin{minipage}{\textwidth}
\begin{itemize}
\item Das Ultraschallgerät/ Sonoscape, auf dem die Ultraschallbilder generiert werden.
\item Das mobile Endgerät, für das eine Anwendung zur Anzeige der Ultraschallbilder implementiert wird.
\item Die Halterung für die Befestigung des mobilen Endgeräts und dem halbdurchlässigen Spiegel an dem Ultraschallkopf.
\end{itemize}
\end{minipage}

\section{Risikoanalyse}
Die nachfolgenden Tabellen \ref{tab:Risikoanalyse} und \ref{tab:Risikomassnahmen} umfassen die identifizierten Risiken und die geplanten Maßnahmen, die bei der Entwicklung und Umsetzung des Projekts auftreten können. Insgesamt wurden 11 Risiken ermittelt, die eine Realisierung des Projekts gefährden. Diese werden in der nachfolgenden Tabelle vorgestellt. Die Tabelle \ref{tab:Risikoanalyse} umfasst eine zugeordnete RisikoID, eine Beschreibung des Risikos sowie den dadurch auftretenden Effekt.

\begin{center}
\begin{table} [H]
    \begin{tabular}{ | p{0.1\textwidth} | p{0.45\textwidth} | p{0.45\textwidth} |}
    \hline
   \textbf{RisikoID} &  \textbf{Risiko} & \textbf{Effekt}\\
      \hline
      1 &	
      Zugriff auf das Betriebssystem vorhanden/ erlangen &	
      Zugriff auf das Betriebssystem wird für die Extraktion der Bilddaten und die Installation von zusätzlicher Software und Treibern benötigt \\
      \hline
      2	&
      Ausreichende Berechtigungen auf dem Ultraschallgerät vorhanden &
      Berechtigungen werden für die Installation von Software und Treibern auf dem Ultraschall benötigt\\
      \hline    
      3&	Zugriff auf Bilddaten des Ultraschallgeräts vorhanden	&Benötigten Ultraschallbilder müssen extrahiert/ aufgenommen werden können\\
      	\hline
4& 	Verfügt das Ultraschallgerät über eine Netzwerk-Schnittstelle &	Die Netzwerkschnittstelle wird für die Übertragung der Bilddaten benötigt\\
      	\hline
5&	Erweiterung des Ultraschallgeräts um eine Netzwerkkomponente	&Die Netzwerkschnittstelle wird für die Übertragung der Bilddaten verwendet\\
      	\hline
6	&Können die Ultraschallbilder im Ursprungsformat in einem gewissen Zeitintervall übertragen werden	&Die Übertragung der Ultraschallbilder in ihrem Ursprungsformat kann sich auf die Framerate (Anzahl der zu übertragenden Bilder) und die Performance auswirken\\
      	\hline
7	&Ausreichend Rechenleistung für die Echtzeitfähigkeit	&Es wird ausreichend Rechenleistung für die Komprimierung und Dekodierung der Bilddaten benötigt, um eine geringe Latenz bzw. die Echtzeitfähigkeit zu gewährleisten.\\
      	\hline
8	&Ausreichend Rechenleistung für die Dekodierung der Bilddaten und die Bildverarbeitung vorhanden	&Es wird ausreichend Rechenleistung für die Dekodierung und die Bildverarbeitung benötigt\\
      	\hline
9	&Ist eine Realitätsgetreue Darstellung der Ultraschallbilder auf dem mobilen Endgerät möglich	&Das mobile Endgerät muss über eine ausreichende Auflösung, Helligkeit und eine geeignete Displaygröße verfügen\\
      	\hline
10	&Kann eine leichtgewichtige, starre und stabile Vorrichtung entworfen und umgesetzt werden	&Aufgrund der exakten Vorgabe der Ausrichtung der einzelnen Komponenten (Abstand, Winkel, Höhe) muss eine Vorrichtung gebaut werden, auf der sich das mobiles Endgerät und der halbdurchlässige Spiegel befestigen lassen\\
      	\hline
11	&Kann die Vorrichtung an dem Ultraschallkopf angebracht werden	&Die Vorrichtung muss  an einer festgelegten Stelle fixiert und befestigt werden um eine Realitätsgetreue Überlagerung der Bilddaten mit der Realität zu ermöglichen \\
\hline
  
    \end{tabular}
     \label{tab:Risikoanalyse}
     \caption{{\small Tabelle Risikoanalyse}}
    \end{table}
\end{center}
Von den ermittelten 11 Risiken haben sieben Risiken so eine gravierende (Tragweite 5) Auswirkung auf die Umsetzung und Realisierung des Projekts, dass diese Risiken eine hohe Priorität hatten und dadurch bereits zu Beginn des Projekts intensiv analysiert wurden, um entsprechende Maßnahmen zu planen. Hierbei ging es vor allem um den Zugriff auf das Ultraschallgerät, die Installation von benötigten Treibern und Software, das Auslesen des Bildspeichers, der Netzwerkzugriff des Ultraschallgeräts und die Konstruktion einer stabilen Halterung, die an dem Ultraschallkopf befestigt werden kann. Die Entsprechenden Maßnahmen reichten hierbei von der Konsultation und Kommunikation mit dem Geräthersteller über die Analyse der beteiligten Komponenten bis hin zur Umsetzung von Prototypen während der Planung und der Realisierung des Projekts.  \\

In der nachfolgenden Tabelle \ref{tab:Risikomassnahmen} sind die zugehörigen Arbeitspakete angegeben in dem die geplante Maßnahme durchgeführt wurde. Jede Maßnahme wird beschrieben, die Eintrittswahrscheinlichkeit angegeben und die Auswirkung durch die Tragweite des Risikos bei einem Eintritt festgelegt. 
Die „Tragweite“ repräsentiert die Auswertung des Risikos, hierbei handelt es sich ebenso um die Priorität der einzelnen Risiken wie auch um die Auswirkung auf die Durchführbarkeit bzw. die Gefährdung der Projektumsetzung (bspw. die Einhaltung der festgelegten Meilensteine, Auswirkung auf die Qualität bzw. die Performance). Je höher der Wert ist umso höher ist die Priorität der Umsetzung der Maßnahme für die Realisierung. Die Skala für die Tragweite ist durch die folgenden fünf Werte und deren Auswirkung definiert:\\
\begin{minipage}{\textwidth}
\begin{itemize}
\item 1 = zu vernachlässigen
\item 2 = unkritisch
\item 3 = in Bereichen kritisch
\item 4 = Kritisch
\item 5 = projektgefährdend
\end{itemize}
\end{minipage}


\begin{center}
\begin{minipage}{0.9\textwidth}
\begin{table} [H]
    \begin{tabular}{ | p{0.1\textwidth} | p{0.35\textwidth} | p{0.1\textwidth} | p{0.25\textwidth} | p{0.15\textwidth} | } 
    \hline
   \textbf{RisikoID} &  \textbf{Maßnahme(n)} & \textbf{Arbeits-paket} & \textbf{Eintritts-wahrscheinlichkeit} & \textbf{Tragweite} \\
      \hline
1	&Konsultation und Kommunikation mit dem Gerätehersteller	&4	&80\%	&5 \\
\hline
2	&Analyse vorhandener bzw. Erweiterung der Berechtigungen des Ultraschallgeräts	&5	&100\%	&5\\
\hline
3	&Analyse der Zugriffsmöglichkeiten auf den internen Bildspeicher oder Installation und Ausführung zusätzlicher Software	&6	&80\%	&5\\
\hline
4	&Vorhandene Schnittstellen des Ultraschallgeräts analysieren. 	&7	&100\%	&5 \\
\hline
5	&Die Möglichkeiten für die  Installation einer zusätzlichen Netzwerkschnittstelle überprüfen (zusätzliche Hardware: Router, WLAN- Stick)	&7	&100\%	&5 \\
\hline
6	&Um den zu übertragenden  Datenumfang zu reduzieren, muss ggf. eine Komprimierung der Daten erfolgen.	&16	&60\%	&3 \\
\hline
7	&Analyse der benötigten Rechenleistung 	&22	&60\%	&2 \\
\hline
8	&Analyse der benötigten Rechenleistung und Beschaffung eines mobilen Endgerätes mit genügend Rechenleistung	&2	&20\%	&2 \\
\hline
9	&Festlegen der benötigten Auflösung und der Displaygröße	&2	&20\%	&1 \\
\hline
10	&Planung der Vorrichtung,
Analyse der benötigten Materialen, Prototyping	&11	&100\%	&5 \\
\hline
11	&Prototyping	&11	&100\%	&5     \\
\hline
     \end{tabular}
     \label{tab:Risikomassnahmen}
     \caption{{\small Tabelle Risikomaßnahmen}}
     \end{table}
     \end{minipage}
\end{center}

\section{Anforderungsanalyse}
Im Folgenden werden die Funktionalen Anforderungen für die einzelnen Bereiche „Sonoscape/ Ultraschallgerät“, „Datenübertragung“, „mobile Applikation“, „Bildverarbeitung“ und die „grafische Benutzeroberfläche“ festgelegt. Hierbei werden die Definition der Funktionalitäten vorgenommen, welche die gesamte Anwendung erfüllen muss (Absolute Gültigkeit), soll (Empfehlung) oder kann (Optional).  
\subsection{Funktionale Anforderungen}
\begin{minipage}{\textwidth}
Sonoscape/ Ultraschallgerät
\begin{itemize}
\item Das erstellte Ultraschallbild muss in der Auflösung extrahiert werden können, in der es vom Ultraschallgerät generiert wurde.
\item Das Ultraschallgerät muss um eine Netzwerkkomponente für den Netzwerkzugriff ergänzt werden.
\item Das Sonoscape muss um eine Serverfunktionalität ergänzt werden über die, die Ultraschallbilder übertragen werden.
\item Die extrahierten Bilder/ Frames des Ultraschallgeräts müssen vor der Übertragung in ein entsprechendes Format konvertiert werden.
\end{itemize}
\end{minipage}

\begin{minipage}{\textwidth}
Datenübertragung
\begin{itemize}
\item Um die zu übertragenden Daten zu verringern, muss eine Datenkomprimierung durchgeführt werden. 
\item Die Datenübertragung muss durch einen IO-Service realisiert werden.
\item Die Ultraschallbilder müssen kontinuierlich separat übertragen werden.
\item Die Ultraschallbilder müssen mit einer konstanten Framerate zw. 12-30 Frames per second (fps) übertragen und angezeigt werden. Die Variabilität der Framerate ist abhängig von der Anzahl der erzeugten Ultraschalbilder durch das Ultraschallgerät. 12 Frames werden von dem Gerät erzeugt, wenn zusätzliche Informationen in den Bilder angezeigt werden (bspw. Doppler Effekt), 30 Frames hingegen werden bei der normalen Verwendung ohne zusätzliche Informationen erzeugt.
\item Das Protokoll für die Datenübertragung muss TCP/IP sein.
\end{itemize}
\end{minipage}

\begin{minipage}{\textwidth}
Mobile Applikation
\begin{itemize}
\item Die mobile Applikation muss eine Verbindung zum Server aufbauen.
\item Die mobile Applikation muss über einen festgelegten Port auf eingehende Daten lauschen.
\item Die mobile Applikation muss die Konvertierung des ankommenden Bytestreams in ein entsprechendes Format durchführen.
\item Die verarbeiteten Frames müssen auf der Benutzeroberfläche angezeigt werden. Anschließend müssen die Frames gelöscht werden und der belegte Speicherplatz wieder freigegeben werden. 
\item Bei der Verwendung der mobilen Applikation kann der Benutzer über einen Verbindungsabbruch oder einen Datenverlust informiert werden.
\end{itemize}
\end{minipage}

\begin{minipage}{\textwidth}
Bildverarbeitung
\begin{itemize}
\item Die Bildverarbeitung muss innerhalb der mobilen Anwendung realisiert werden.
\item Die Bildverarbeitung umfasst das Ausschneiden der relevanten Informationen des Ultraschallbildes und die Zusammensetzung ein neuen Frames.
\end{itemize}
\end{minipage}

\begin{minipage}{\textwidth}
Die grafische Benutzeroberfläche
\begin{itemize}
\item Die Benutzeroberfläche der mobilen Applikation muss einheitlich und simpel aufgebaut sowie intuitiv benutzbar sein, um bei der Verwendung während einer Ultraschalluntersuchung nicht abzulenken.
\item Auf der grafischen Benutzeroberfläche müssen die Ultraschallbilder kontinuierlich angezeigt werden.
\end{itemize}
\end{minipage}
\subsection{Nichtfunktionale Anforderungen}
\begin{minipage}{\textwidth}
Mobiles Endgerät
\begin{itemize}
\item Die Lichtdichte (Helligkeit) des Displays muss min. 500 cd/m² betragen.
\item Die Auflösung muss min. 1920x 1080 (Full HD) betragen.
\item Die Größe des Displays muss min. 5.0 Zoll betragen.
\item Das Gewicht darf höchstens 200 Gramm betragen.
\item Das Gerät muss über eine Netzwerkschnittstelle (WLAN (802.11)) verfügen.
\item Das Gerät muss über einen Beschleunigungssensor und ein Gyroskop verfügen.
\item Das Gerät muss über ausreichend Rechenleistung für die Bilddecodierung, Bildverarbeitung, Bildanzeige und min. 4 Kerne (Quad-Core) verfügen.
\item Das Gerät muss auf einem offenen Betriebssystem basieren, für das unentgeltlich Programme implementiert und installiert werden können.
\end{itemize}
\end{minipage}

\begin{minipage}{\textwidth}
Halterung
\begin{itemize}
\item Es müssen nahezu identische Lichtverhältnisse auf beiden Seiten des halbdurchlässigen Spiegels existieren. Evtl. kann dazu eine zusätzliche Lichtquelle installiert werden.
\item Das Gewicht der kompletten Halterung (mobiles Endgerät, halbdurchlässiger Spiegel, Material zur Befestigung)  soll 400 Gramm nicht überschreiten.
\item Um den festgelegten Winkel zw. mobilem Endgerät, Auge des Betrachters und dem Objekt nicht zu verändern muss ein starres Material verwendet werden.
\end{itemize}
\end{minipage}

\begin{minipage}{\textwidth}
Darstellung der Ultraschallbilder
\begin{itemize}
\item Da nicht alle dargestellten Informationen des Ultraschallgerätes auf dem Ursprungsbildes benötigt werden, müssen die benötigten Informationen aus dem Ursprungsbild ausgeschnitten und zu einem neuen Bild zusammengefügt werden.
\item Um die komplette Darstellung des Ultraschallbildes zu erfassen muss die Zoomstufe eins auf dem Ultraschallgerät ausgewählt sein.
\end{itemize}
\end{minipage}
