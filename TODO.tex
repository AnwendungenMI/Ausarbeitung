\chapter{TODO}
Restliche Hardware beschaffen (Linux-Gerät adapter Stromversorgung usw.)\\
Anfänglichen Plan aufarbeiten (Arbeitspakete einteilen usw.)\\
Inhaltsverzeichnis besprechen\\
Präzi Präsentation\\
Prototypen bauen
\begin{enumerate}
\item{Skizze verstehen}
\item{Theoretisch überlegen wie Prototyp realisiert werden kann(stützstreben, halterung handy, halterung ultraschallkopf, halterung spiegel, bohren, kleben, halterung zu schwer für hand ?! usw. )}
\item{wo kann das ganze gebacken / gelagert werden}
\item eventuell erste testläufe für bestimmte teile (übung)
\item materialbeschaffung
\item Realisierung des prototypen
\end{enumerate}
Validierung anfertigen, was genau soll validiert werden?\\
--> handybild korrekt über spiegel im material / objekt abgebildet ?
\begin{enumerate}
\item{bestimmte testfälle entwerfen}
\item "Test-Block-Masse" vorbereiten und entsprechende Gegenstände einbetten
\item fertigen prototypen nutzen um die tests durchzuführen, dabei 
bilder / videos machen um material für die schriftliche arbeit zu generieren
\end{enumerate}
Abschlussarbeit schreiben\\
Abschlussarbeit zusammenfügen\\
Vorführung vorbereiten, sprich Gerät kontrollieren, alles was nötig ist bereitlegen, ablauf überlegen falls erforderlich\\
Präsentation halten
LED